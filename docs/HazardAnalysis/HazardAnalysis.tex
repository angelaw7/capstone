\documentclass{article}

\usepackage{booktabs}
\usepackage{tabularx}
\usepackage{hyperref}
\usepackage[hmargin=3cm]{geometry}
\usepackage{pdflscape}
\usepackage{longtable}


\hypersetup{
    colorlinks=true,       % false: boxed links; true: colored links
    linkcolor=red,          % color of internal links (change box color with linkbordercolor)
    citecolor=green,        % color of links to bibliography
    filecolor=magenta,      % color of file links
    urlcolor=cyan           % color of external links
}

\title{Hazard Analysis\\\progname}

\author{\authname}

\date{}

\input{../Comments}
%% Common Parts

\newcommand{\progname}{Plutos} % PUT YOUR PROGRAM NAME HERE
\newcommand{\authname}{Team \#10, Plutos
\\ Jason Tan
\\ Payton Chan
\\ Eric Chen
\\ Fondson Lu
\\ Angela Wang} % AUTHOR NAMES                  

\usepackage{hyperref}
    \hypersetup{colorlinks=true, linkcolor=blue, citecolor=blue, filecolor=blue,
                urlcolor=blue, unicode=false}
    \urlstyle{same}
                                


\begin{document}

\maketitle
\thispagestyle{empty}

~\newpage

\pagenumbering{roman}

\begin{table}[hp]
\caption{Revision History} \label{TblRevisionHistory}
\begin{tabularx}{\textwidth}{llX}
\toprule
\textbf{Date} & \textbf{Developer(s)} & \textbf{Change}\\
\midrule
10/23/2024 & Angela & Initial draft\\
... & ... & ...\\
\bottomrule
\end{tabularx}
\end{table}

~\newpage

\tableofcontents

~\newpage

\pagenumbering{arabic}

\section{Introduction}

A hazard in the context of this document is any property or condition that may
lead to harm or damage to the Plutos system or its users. Potential losses due
to these hazards may include loss of application functionality, performance, or
accuracy, or breaches of user privacy or data. The following sections will
identify hazards within the system and discuss the controls in place for their
mitigation.


\section{Scope and Purpose of Hazard Analysis}

This document aims to provide a comprehensive hazard analysis of the Plutos
system. It identifies hazards within the system, outlines measures to mitigate
them, and specifies the safety and security requirements derived from this
analysis. The analysis will follow the Failure Mode and Effect Analysis (FMEA)
approach. The analysis aims to discover the potential failure modes within the
system and develop a mitigation plan to reduce the risk of failure. 


\section{System Boundaries and Components}

The system will be divided into the following components:
\begin{enumerate}
    \item The Plutos application, which consists of:
    \begin{enumerate}
        \item \textbf{The database}: The database is where the user’s receipts
        and profile data will be stored.
        \item \textbf{The backend server}: The backend server is responsible for
        handling and serving requests from the client. It will interact with all
        the other components listed here.
        \item \textbf{The frontend/user interface}: The frontend/user interface
        is responsible for displaying the appropriate views to the user and
        handling user interactions.
        \item \textbf{The machine learning (ML) model}: The ML model is
        responsible for parsing and categorizing items from a picture of an
        itemized receipt.
    \end{enumerate}
    \item The user’s mobile device and camera setup
\end{enumerate}


\section{Critical Assumptions}

The project will be making the following critical assumptions: 
\begin{enumerate}
    \item The users will be
    using a mobile device running an up-to-date version of iOS or Android.
    \item Users are
    not expected to repeatedly input invalid images into the system (i.e., images
    that do not contain a receipt). While it is anticipated that users may
    occasionally submit an invalid image, it is assumed to not be a significant
    concern.
\end{enumerate}  


\newgeometry{bottom=25mm,hmargin=1.5cm,vmargin=3cm, landscape}
\begin{landscape}

\section{Failure Mode and Effect Analysis}

\wss{Include your FMEA table here. This is the most important part of this document.}
\wss{The safety requirements in the table do not have to have the prefix SR.
The most important thing is to show traceability to your SRS. You might trace to
requirements you have already written, or you might need to add new
requirements.}
\wss{If no safety requirement can be devised, other mitigation strategies can be
entered in the table, including strategies involving providing additional
documentation, and/or test cases.}

    \begin{longtable}{|l|l|l|l|l|l|l|}
        \caption{Failure Mode and Effect Analysis Table} \label{tab:long} \\
        
        \hline
        \multicolumn{1}{|c|}{\textbf{Design Function}} & \multicolumn{1}{c|}{\textbf{Failure Modes}} & \multicolumn{1}{c|}{\textbf{Effects of Failure}} & \multicolumn{1}{c|}{\textbf{Causes of Failure}} & \multicolumn{1}{c|}{\textbf{Recommended Action}} & \multicolumn{1}{c|}{\textbf{SR}} & \multicolumn{1}{c|}{\textbf{Ref}}\\
        \hline\endfirsthead\hline\endlastfoot
        
        ... & ... & ... & ... & ... & ... & ...\\
        \end{longtable}
    \newpage{}
\end{landscape}
\restoregeometry

\section{Safety and Security Requirements}

\wss{Newly discovered requirements.  These should also be added to the SRS.  (A
rationale design process how and why to fake it.)}

\section{Roadmap}

\wss{Which safety requirements will be implemented as part of the capstone timeline?
Which requirements will be implemented in the future?}

\newpage{}

\section*{Appendix --- Reflection}

\begin {enumerate}
\item \textbf{Why is it important to create a development plan prior to starting the
Project?}

Creating a development plan before starting the project is crucial so the whole team can discuss/agree upon the key project details and scope. It is vital to lay out the groundwork for our project and define the direction needed to achieve our goals optimally. We found it especially important to discuss team dynamics – specifying meeting details, expectations from each member of the team, and the general workflow plan. Defining these elements upfront helps keep everyone aligned and organized before we dive into the project details and implementation. Breaking down the project into high-level milestones allows us to create well-defined and achievable timelines to guide our progress.

\item \textbf{In your opinion, what are the advantages and disadvantages of using CI/CD?}

We believe that CI/CD is a great tool due to it allowing automation and quality control within our development workflow. This significantly reduces manual testing labour and saves time which will be crucial for the development of the Plutos app. We aim to at least include running tests, linters, and formatters within our CI pipeline so that we may be confident that all changes made meet a certain quality level. One drawback that we will need to be aware of is that the setup of the CI/CD pipeline may pose a challenge due to the team’s lack of experience in setting up such a workflow. Having too much automation could lead us to have false confidence in our code, especially if our test coverage is not meeting standards due to rapid development. It’s important not to rely too heavily on the CI/CD pipeline and understand that it’s a tool that is meant to help gauge the overall quality of our code and not something that will replace manual testing.

\item \textbf{What disagreements did your group have in this deliverable, if any, and how did you resolve them?}

Our team had differing opinions when we discussed the timeline of our project. Some members felt we could reduce the time allocated for end-to-end testing from five weeks to three, while others preferred to extend the frontend and backend development by an additional week. We also had discussions for how much buffer time was needed to account for potential delays. After some discussion, we came to an agreement that the frontend and backend could be done in parallel, allowing us to dedicate 6 weeks to both sections. These differences were ultimately resolved through mutual understanding, as many of us would be unavailable during exam season in December and January. What ultimately helped us resolve our differences was the fact that we all remained realistic about potential challenges we might face in the future.

\end{enumerate}

\begin{enumerate}

\item What went well while writing this deliverable?

The overall process while writing this deliverable was smooth and efficient as we were quickly able to identify the potential hazards related to our project. We brainstormed several ambiguous sections or things we thought were a bit unclear within this analysis document, and were able to get very clear answers from our helpful TA. The team worked well together as we all put in our best efforts and supported one another when completing this task. 

Using the FMEA (Failure Mode and Effect Analysis) approach helped streamline the hazard identification process. Breaking down the system into components allowed for a clear understanding of where risks might occur. Writing the deliverable helped the team clarify and solidify their understanding of how the receipt scanner, the AI model, and other system components interact, making it easier to identify hazards.

\item What pain points did you experience during this deliverable, and how did you resolve them?

At first, it was challenging to define the potential failure modes, especially for components like the machine learning model. The team resolved this by conducting additional research on common failure points in similar systems and reviewing how AI models typically behave with poor input data. Another challenge was balancing realistic assumptions about user behavior with potential risks. For example, while assuming users won’t repeatedly input invalid images, we acknowledged this could still happen. We resolved this by planning mitigation strategies for those edge cases.

\item Which of your listed risks had your team thought of before this deliverable, and which did you think of while doing this deliverable? For the latter ones (ones you thought of while doing the Hazard Analysis), how did they come about?

We had already considered risks related to image quality (e.g., blurry or incomplete receipt images) and network connectivity issues (e.g., users not being able to connect to the database).

While working on the hazard analysis, we realized potential risks like Optical Character Recognition (OCR) misinterpretation due to varied receipt fonts and ML model processing time under different device conditions (e.g., low memory or poor network). These came up while brainstorming as a team and thinking about the specific steps in image processing and how the system handles diverse input.

\item Other than the risk of physical harm, list at least 2 other types of risk in software products. Why are they important to consider?

Two other risks that are apparent in software products are security and reliability risks. 

Security vulnerabilities can lead to issues such as data breaches, unauthorized access or identity theft, as well as collateral damages, whether it be financial losses or reputational damage. This is considered a risk and is important to consider as it creates an opportunity for malicious users to exploit weaknesses in software systems, which can have a range of detrimental consequences. Examples include operational disruptions, intellectual property theft, ransomware attacks, etc.

As for reliability, it is mostly concerned with when software fails to function consistently, such as having frequent downtimes. This can affect the user’s experience, leading to a loss of productivity or customer dissatisfaction. Both of these can lead to potential revenue loss. This is classified as a risk and is important to consider because unreliable software can lead to negative consequences, which affect not only the users but also the organization that provides the software. The damages can be both monetary and non-monetary, such as losing user trust/loyalty, reputational damage, and associated compliance and legal risks.
\end{enumerate}

\end{document}