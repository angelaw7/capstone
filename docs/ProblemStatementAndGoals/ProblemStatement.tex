\documentclass{article}

\usepackage{tabularx}
\usepackage{booktabs}

\title{Problem Statement and Goals\\\progname}

\author{\authname}

\date{}

\input{../Comments}
%% Common Parts

\newcommand{\progname}{Plutos} % PUT YOUR PROGRAM NAME HERE
\newcommand{\authname}{Team \#10, Plutos
\\ Jason Tan
\\ Payton Chan
\\ Eric Chen
\\ Fondson Lu
\\ Angela Wang} % AUTHOR NAMES                  

\usepackage{hyperref}
    \hypersetup{colorlinks=true, linkcolor=blue, citecolor=blue, filecolor=blue,
                urlcolor=blue, unicode=false}
    \urlstyle{same}
                                


\begin{document}

\maketitle

\begin{table}[hp]
\caption{Revision History} \label{TblRevisionHistory}
\begin{tabularx}{\textwidth}{llX}
\toprule
\textbf{Date} & \textbf{Developer(s)} & \textbf{Change}\\
\midrule
09/18/2024 & Angela Wang & Initial Draft\\
09/20/2024 & Jason Tan & Environment section and touch ups\\
... & ... & ...\\
\bottomrule
\end{tabularx}
\end{table}

\section{Problem Statement}

% \wss{You should check your problem statement with the
% \href{https://github.com/smiths/capTemplate/blob/main/docs/Checklists/ProbState-Checklist.pdf}
% {problem statement checklist}.} 

% \wss{You can change the section headings, as long as you include the required
% information.}

\subsection{Problem}

Young adults often face challenges in managing their finances effectively,
especially when it comes to tracking expenses and budgeting. For young adults
tracking expenses and maintaining a budget can feel too cumbersome, leading many
to avoid the process altogether. 

Despite advancements in AI and automation, most budgeting apps still lack
automatic expense tracking, making the process feel tedious and inefficient.
This creates frustration and prevents young adults from leveraging modern
technology to gain control over their finances and develop better spending
habits.


\subsection{Inputs and Outputs}

% \wss{Characterize the problem in terms of ``high level'' inputs and outputs.  
% Use abstraction so that you can avoid details.}

\textbf{Inputs:} User inputs data about their expenses and budget goals.\\
\textbf{Outputs:} Visualizations of the user's spending allocations in
comparison to their set budget, and recommendations for how they may adjust
their spending to meet their goals.

\subsection{Stakeholders}

Stakeholders include anyone who is looking to better manage their finances, set
budget goals, and track their spending, with a focus on first and second-year
university students who are just starting to live on their own.

\subsection{Environment}

% \wss{Hardware and software environment}

\textbf{Software:} iPhone, Android, iPad\\
\textbf{Hardware:} Mobile devices with a camera

\section{Goals}

Our goals include the following:
\begin{itemize}
    \item Develop a machine learning model that can accurately
    (\textgreater90\%) parse commonly purchased items (e.g., groceries, cleaning
    supplies) from a picture of a receipt.
    \item Develop a machine learning model that can accurately
    (\textgreater90\%) categorize items into approprite, pre-defined spending
    categories. 
    \item Develop a mobile application that allows users to take a picture of
    their receipt or manually input their expenses. These expenses would be
    stored in a database so that the user can review their spending history.
    \item Within the application, display visualizations of the user's purchases
    and spending allocations, and provide recommendations for how they may
    adjust their spending to meet their budget goals. These should be catered to
    the user's personal spending habits and goals.
    \item Allow users to set budget goals over different time intervals (e.g.,
    short-term, long-term) and track their progress towards these goals.
\end{itemize}

\section{Stretch Goals}
\begin{itemize}
    \item Build upon the base machine learning model to train on the user's
    personal spending data to provide more accurate item parsing and
    categorization (i.e., for items with similar names, the model can learn
    which category the user typically assigns them to).
    \item Build upon the base machine learning model to categorize items using
    user-defined/customizable spending categories.
    \item Build upon the base machine learning model to predict future spending
    based on the user's spending history and provide recommendations for how
    they can adjust their spending to meet their budget goals.
    \item Gamify the application to make it more engaging and encourate users to
    meet their budget goals and develop better spending habits.
    \item Allow users to input expenses through speech recognition, where the
    application can parse the user's speech and categorize the items
    accordingly.
\end{itemize}

\section{Challenge Level and Extras}

% \wss{State your expected challenge level (advanced, general or basic).  The
% challenge can come through the required domain knowledge, the implementation
% or something else.  Usually the greater the novelty of a project the greater
% its challenge level.  You should include your rationale for the selected
% level. Approval of the level will be part of the discussion with the
% instructor for approving the project.  The challenge level, with the approval
% (or request) of the instructor, can be modified over the course of the term.}

% \wss{Teams may wish to include extras as either potential bonus grades, or to
% make up for a less advanced challenge level.  Potential extras include
% usability testing, code walkthroughs, user documentation, formal proof,
% GenderMag personas, Design Thinking, etc.  Normally the maximum number of
% extras will be two.  Approval of the extras will be part of the discussion
% with the instructor for approving the project.  The extras, with the approval
% (or request) of the instructor, can be modified over the course of the term.}

The expected challenge level is \textbf{general}. The primary challenge of the
project is developing a machine learning model that can accurately parse items
from a picture of a receipt, and to categorize them into appropriate spending
categories. This requires a strong understanding of training and tuning models
on image data to achieve high accuracy. Additionally, different items across
various stores may have similar names or be difficult to recognize, which adds
to the complexity of the task. The other part of the project is to develop a
user-friendly mobile application, which is a more general software engineering
component.

Furthermore, our team plans to include the following two extras:
\begin{itemize}
    \item \textbf{Requirements elicitation}: We will conduct interviews and a
    survey to gather requirements from potential users to determine user needs
    and preferences.
    \item \textbf{Usability testing}: We will ask potential users to test the
    application and provide feedback on its usability and functionality.

\end{itemize}

\newpage{}

\section*{Appendix --- Reflection}

\begin {enumerate}
\item \textbf{Why is it important to create a development plan prior to starting the
Project?}

Creating a development plan before starting the project is crucial so the whole team can discuss/agree upon the key project details and scope. It is vital to lay out the groundwork for our project and define the direction needed to achieve our goals optimally. We found it especially important to discuss team dynamics – specifying meeting details, expectations from each member of the team, and the general workflow plan. Defining these elements upfront helps keep everyone aligned and organized before we dive into the project details and implementation. Breaking down the project into high-level milestones allows us to create well-defined and achievable timelines to guide our progress.

\item \textbf{In your opinion, what are the advantages and disadvantages of using CI/CD?}

We believe that CI/CD is a great tool due to it allowing automation and quality control within our development workflow. This significantly reduces manual testing labour and saves time which will be crucial for the development of the Plutos app. We aim to at least include running tests, linters, and formatters within our CI pipeline so that we may be confident that all changes made meet a certain quality level. One drawback that we will need to be aware of is that the setup of the CI/CD pipeline may pose a challenge due to the team’s lack of experience in setting up such a workflow. Having too much automation could lead us to have false confidence in our code, especially if our test coverage is not meeting standards due to rapid development. It’s important not to rely too heavily on the CI/CD pipeline and understand that it’s a tool that is meant to help gauge the overall quality of our code and not something that will replace manual testing.

\item \textbf{What disagreements did your group have in this deliverable, if any, and how did you resolve them?}

Our team had differing opinions when we discussed the timeline of our project. Some members felt we could reduce the time allocated for end-to-end testing from five weeks to three, while others preferred to extend the frontend and backend development by an additional week. We also had discussions for how much buffer time was needed to account for potential delays. After some discussion, we came to an agreement that the frontend and backend could be done in parallel, allowing us to dedicate 6 weeks to both sections. These differences were ultimately resolved through mutual understanding, as many of us would be unavailable during exam season in December and January. What ultimately helped us resolve our differences was the fact that we all remained realistic about potential challenges we might face in the future.

\end{enumerate}

\begin{enumerate}
    \item What went well while writing this deliverable? 
    \item What pain points did you experience during this deliverable, and how
    did you resolve them?
    \item How did you and your team adjust the scope of your goals to ensure
    they are suitable for a Capstone project (not overly ambitious but also of
    appropriate complexity for a senior design project)?
\end{enumerate}  

\end{document}