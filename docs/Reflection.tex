\begin {enumerate}
\item \textbf{Why is it important to create a development plan prior to starting the
Project?}

Creating a development plan before starting the project is crucial so the whole team can discuss/agree upon the key project details and scope. It is vital to lay out the groundwork for our project and define the direction needed to achieve our goals optimally. We found it especially important to discuss team dynamics – specifying meeting details, expectations from each member of the team, and the general workflow plan. Defining these elements upfront helps keep everyone aligned and organized before we dive into the project details and implementation. Breaking down the project into high-level milestones allows us to create well-defined and achievable timelines to guide our progress.

\item \textbf{In your opinion, what are the advantages and disadvantages of using CI/CD?}

We believe that CI/CD is a great tool due to it allowing automation and quality control within our development workflow. This significantly reduces manual testing labour and saves time which will be crucial for the development of the Plutos app. We aim to at least include running tests, linters, and formatters within our CI pipeline so that we may be confident that all changes made meet a certain quality level. One drawback that we will need to be aware of is that the setup of the CI/CD pipeline may pose a challenge due to the team’s lack of experience in setting up such a workflow. Having too much automation could lead us to have false confidence in our code, especially if our test coverage is not meeting standards due to rapid development. It’s important not to rely too heavily on the CI/CD pipeline and understand that it’s a tool that is meant to help gauge the overall quality of our code and not something that will replace manual testing.

\item \textbf{What disagreements did your group have in this deliverable, if any, and how did you resolve them?}

Our team had differing opinions when we discussed the timeline of our project. Some members felt we could reduce the time allocated for end-to-end testing from five weeks to three, while others preferred to extend the frontend and backend development by an additional week. We also had discussions for how much buffer time was needed to account for potential delays. After some discussion, we came to an agreement that the frontend and backend could be done in parallel, allowing us to dedicate 6 weeks to both sections. These differences were ultimately resolved through mutual understanding, as many of us would be unavailable during exam season in December and January. What ultimately helped us resolve our differences was the fact that we all remained realistic about potential challenges we might face in the future.

\end{enumerate}