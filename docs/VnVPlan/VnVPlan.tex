\documentclass[12pt, titlepage]{article}

\usepackage{booktabs}
\usepackage{tabularx}
\usepackage{hyperref}
\hypersetup{
    colorlinks,
    citecolor=blue,
    filecolor=black,
    linkcolor=red,
    urlcolor=blue
}
\usepackage[round]{natbib}

\input{../Comments}
%% Common Parts

\newcommand{\progname}{Plutos} % PUT YOUR PROGRAM NAME HERE
\newcommand{\authname}{Team \#10, Plutos
\\ Jason Tan
\\ Payton Chan
\\ Eric Chen
\\ Fondson Lu
\\ Angela Wang} % AUTHOR NAMES                  

\usepackage{hyperref}
    \hypersetup{colorlinks=true, linkcolor=blue, citecolor=blue, filecolor=blue,
                urlcolor=blue, unicode=false}
    \urlstyle{same}
                                


\begin{document}

\title{System Verification and Validation Plan for \progname{}} 
\author{\authname}
\date{\today}
	
\maketitle

\pagenumbering{roman}

\section*{Revision History}

\begin{tabularx}{\textwidth}{p{3cm}p{2cm}X}
\toprule {\bf Date} & {\bf Version} & {\bf Notes}\\
\midrule
11/01/2024 & 0.1 & Add: 3.1, 3.2, 3.3, 3.6\\
11/02/2024 & 0.2 & Update Rev0 draft\\
11/03/2024 & 0.3 & Add 4.1 for Rev0\\
... & ... & ...\\
\bottomrule
\end{tabularx}

~\\
\wss{The intention of the VnV plan is to increase confidence in the software.
However, this does not mean listing every verification and validation technique
that has ever been devised.  The VnV plan should also be a \textbf{feasible}
plan. Execution of the plan should be possible with the time and team available.
If the full plan cannot be completed during the time available, it can either be
modified to ``fake it'', or a better solution is to add a section describing
what work has been completed and what work is still planned for the future.}

\wss{The VnV plan is typically started after the requirements stage, but before
the design stage.  This means that the sections related to unit testing cannot
initially be completed.  The sections will be filled in after the design stage
is complete.  the final version of the VnV plan should have all sections filled
in.}

\newpage

\tableofcontents

\newpage

\section{Symbols, Abbreviations, and Acronyms}

Refer to Section 1.3 of the Software Requirements Specification (SRS) document
for the list of symbols, abbreviations, and acronyms.

In addition, the following abbreviations are used in this document:\\

\renewcommand{\arraystretch}{1.2}
\begin{tabular}{l l} 
  \toprule		
  \textbf{symbol} & \textbf{description}\\
  \midrule 
  T & Test\\
  V\&V & Verification and Validation\\
  \bottomrule
\end{tabular}\\

\newpage

\pagenumbering{arabic}

This document will provide a detailed plan for the Verification and Validation
(V\&V) of Plutos. Below is an overview of what will be covered in this document:
\begin{itemize}
	\item \hyperlink{section.2}{2 General Information}: This section will provide a brief overview of
	the software being tested, its objectives, challenge level, and extras.
	\item \hyperlink{section.3}{3 Plan}: This section will outline the V\&V team roles, the delegation
	of roles to team members, and the V\&V plan.
	\item \hyperlink{section.4}{4 System Tests}: This section will detail the system tests for
	functional and nonfunctional requirements, and traceability between test
	cases and requirements.
	\item \hyperlink{section.5}{5 Unit Test Description}: This section will provide an overview of the
	unit testing scope, unit tests for functional and nonfunctional
	requirements, and traceability between unit tests and modules.
\end{itemize}


\newpage

\section{General Information}

This section will outline the summary of the V\&V document, the objectives of
the process, the challenge level and extras of the project, and the relevant
external documentation that will be referenced in this document.

\subsection{Summary}

This V\&V document outlines the testing approach for Plutos to ensure the
application meets the functional and non-functional requirements outlined in the
SRS document. The verification process will confirm the functionality of
components such as receipt scanning, item recognition, and categorization
function as intended, while validation efforts will focus on ensuring usability,
performance, and security standards are met for an optimal user experience.


\subsection{Objectives}

This subsection will provide an overview of the V\&V objectives that are
in-scope and out-of-scope for the Plutos application.

\subsubsection{In-Scope V\&V Objectives}

The following are the V\&V objectives, outlining what is intended to be verified
or validated as part of the V\&V process: 
\begin{itemize}
	\item \textbf{Accurate Data Extraction and Categorization}: Ensure that the
machine learning (ML) model accurately extracts key information from receipts
and correctly categorizes it by expense type. This is critical to providing
users with precise budgeting insights. The accuracy requirements are denoted in
the Software Requirements Specification (SRS).
	\item \textbf{Seamless User Experience}: Prioritize usability by designing
an intuitive and responsive interface that simplifies the process of scanning
and categorizing receipts. The app should be easy to navigate, allowing users to
view spending summaries effortlessly. 
	\item \textbf{Real-Time Budgeting Feedback} Provide timely updates on
spending habits to make users aware of their financial status. This feature will
empower users to make informed spending decisions based on real-time data.
	\item \textbf{Security and Privacy} Securely handle all user data, with
particular attention to sensitive financial information. This objective includes
encrypting stored data and ensuring compliance with relevant data protection
regulations. 
\end{itemize}

\subsubsection{Out-of-Scope V\&V Objectives}

The following are the V\&V objectives that are out-of-scope due to time or
resource constraints:
\begin{itemize}
	\item \textbf{In-Depth Usability Testing with All User Types} While basic
usability will be validated, comprehensive testing with diverse user
demographics and accessibility concerns is out of scope due to resource
limitations. Feedback gathered post-launch may help inform future usability
improvements. 
	\item \textbf{Verification of External Libraries} The application will use
external libraries for image processing and data handling, assuming these
libraries have been rigorously tested by their developers. We will prioritize
testing our AI model and app functionality rather than the underlying libraries
to focus resources on core functionality. 
	\item \textbf{Accuracy for Poor-Quality Receipts and Non-Common Items}:
Currently, the receipt scanner focuses on high-quality receipts containing
commonly purchased items for students. While expanding the application to
support lower-quality receipts and less common items is planned for future
phases, this enhancement is out of scope for now due to the complexity of
training the ML model during our timeline.

\end{itemize}



\subsection{Challenge Level and Extras}

The challenge level for the project is \textbf{general}. The extras that will be included are: 
\begin{itemize}
	\item \textbf{Requirements elicitation}: Surveys and interviews have been conducted
	as part of developing our Software Requriements Specfication (SRS).
	\item \textbf{Usability testing}: As part of our validation plan, we will conduct
	usability testing in which users will interact with the Plutos application and
	evaluate their experience.
\end{itemize}

\subsection{Relevant Documentation}

The following documents are relevant to the V\&V of Plutos:
\begin{itemize}
	\item
	\href{https://github.com/PlutosCapstone/Plutos/blob/main/docs/ProblemStatementAndGoals/ProblemStatement.pdf}{\textbf{Problem
	Statement and Goals document}}: This document outlines the problem statement
	and goals of the Plutos application, which will considered to ensure the
	V\&V process aligns with the project's purpose and objectives.
	\item
	\href{https://github.com/PlutosCapstone/Plutos/blob/main/docs/SRS/SRS.pdf}{\textbf{Software
	Requirements Specification (SRS) document}}: This document outlines the
	functional and non-functional requirements of the Plutos application,
	serving as the basis for all testing activities.
	\item
	\href{https://github.com/PlutosCapstone/Plutos/blob/main/docs/HazardAnalysis/HazardAnalysis.pdf}{\textbf{Hazard
	Analysis document}}: This document outlines potential hazards and risks
	associated with the Plutos application, which will be considered during V\&V
	to ensure effective risk mitigation.
	\item
	\href{https://github.com/PlutosCapstone/Plutos/blob/main/docs/Design/SoftDetailedDes/MIS.pdf}{\textbf{Module
	Interface Specification (MIS) document}}: This document provides a detailed
	design of the Plutos application, which will be used to inform the
	development of unit tests for each module. \textit{Note: This document is
	yet to be completed as of the time of writing this initial V\&V plan.}
	\item
	\href{https://github.com/PlutosCapstone/Plutos/blob/main/docs/Design/SoftArchitecture/MG.pdf}{\textbf{Module
	Guide (MG) document}}: This document provides a detailed description of the
	software architecture and modules of Plutos, which will guide the
	development of unit tests for each module. \textit{Note: This document is
	yet to be completed as of the time of writing this initial V\&V plan.}
\end{itemize}

\newpage

\section{Plan}

This section introduce the team members involved in the V\&V process and
describe the V\&V plan for the following components:
\begin{itemize}
	\item The SRS
	\item The design
	\item The V\&V
	\item The implementation
\end{itemize}
Following this, there will be a subssections detailing the automated testing and
verification tools, as well as the software validation plan.


\newpage
\subsection{Verification and Validation Team}

There will be a total of 7 distinct roles when it comes to our verification and validation team.
These roles will be split amongst team members and members may share the same secondary roles to facilitate a broader range of tests without overwhelming one person.
This division allows each team member to specialize in a certain aspect of our app when it comes to verification and validation. 
Attached below is a list of roles available in our V\&V group along with the descriptions of what each role entails. \\

\noindent \textbf{V\&V Roles}
\begin{itemize}
	\item Test Case Designer
	\begin{itemize}
		\item Develops detailed test cases for functional requirements, focusing on different aspects such as image quality, data parsing accuracy, and categorization correctness.
		\item Works closely with the other developers to understand key components of the app and to ensure that test cases cover all scenarios, including edge cases.
	\end{itemize}
	\item Quality Assurance Engineer
	\begin{itemize}
		\item Ensures overall quality of the app by performing functional, integration, and system tests.
		\item Coordinates end-to-end testing for receipt scanning and expense categorization workflows.
		\item Collaborates with the Test Case Designer to verify that all test cases have been executed and outcomes meet expected results.
	\end{itemize}
	\item Ai Model Validator
	\begin{itemize}
		\item Focuses on testing the AI model specifically, evaluating its accuracy in parsing text and categorizing expenses.
		\item Conducts model performance assessments under various conditions, such as diverse receipt layouts, lighting conditions, and languages if applicable.
		\item Continuously monitors for model drift and helps recalibrate the model if accuracy degrades over time.
	\end{itemize}
	\item Usability Tester
	\begin{itemize}
		\item Evaluates the app's user interface and user experience for intuitiveness and ease of use.
		\item Conducts usability testing sessions with other students or users in the target demographic (e.g., university students).
		\item Provides feedback on the app's functionality and ensures that user feedback is incorporated into iterative improvements.
	\end{itemize}
	\item Data Integrity Specialist
	\begin{itemize}
		\item Ensures the accuracy and security of the data collected, especially sensitive information such as financial data on receipts.
		\item Verifies that the app correctly anonymizes data if required and that categorization matches expected outputs based on provided datasets.
		\item Works closely with the AI Model Validator to confirm data handling complies with privacy regulations and standards.
	\end{itemize}
	\item Automation Engineer
	\begin{itemize}
		\item Develops automated testing scripts to streamline regression testing and ensure the app functions consistently across different updates.
		\item Automates repetitive tests, especially those related to scanning, categorization, and UI testing, to save time during development sprints.
		\item Coordinates with the Test Case Designer to convert key test cases into automated tests for efficient reuse.
	\end{itemize}
	\item Performance Tester
	\begin{itemize}
		\item Measures and optimizes the app's performance, focusing on response time, load time, and battery usage (important for a mobile app).
		\item Conducts stress tests to determine the app's behavior under heavy usage (e.g., scanning multiple receipts in a short period).
		\item Collaborates with developers to troubleshoot and resolve performance bottlenecks.
	\end{itemize}
\end{itemize}

\noindent The following section delegates the roles above to each team member. Note that each team member has a secondary role which may be repeated amongst other members. The secondary
role is to provide additional assistance when it comes to verification and validation and also serves to bring additional confidence that the system is working
as it should. \\

\noindent \textbf{Delegation of V\&V Roles} \\

\hrule
\vspace{10pt}

\textbf{Team Member 1}
\begin{itemize}
	\item Role(s)
	\begin{enumerate}
		\item Test Case Designer
		\item Quality Assurance Engineer
	\end{enumerate}
	\item Responsibilities
	\begin{enumerate}
		\item Creates comprehensive test cases for functional requirements.
		\item Executes tests to ensure all functionalities are covered.
	\end{enumerate}
\end{itemize}

\hrule
\vspace{10pt}

\textbf{Team Member 2}
\begin{itemize}
	\item Role(s)
	\begin{enumerate}
		\item AI Model Validator
		\item Data Integrity Specialist
	\end{enumerate}
	\item Responsibilities
	\begin{enumerate}
		\item Tests the AI model's accuracy in parsing receipts.
		\item Ensures data handling complies with security and privacy standards.
	\end{enumerate}
\end{itemize}

\hrule
\vspace{10pt}

\newpage

\textbf{Team Member 3}
\begin{itemize}
	\item Role(s)
	\begin{enumerate}
		\item Usability Tester
		\item Quality Assurance Engineer
	\end{enumerate}
	\item Responsibilities
	\begin{enumerate}
		\item Conducts usability tests and collects user feedback.
		\item Supports quality assurance by testing overall workflows and integration.
	\end{enumerate}
\end{itemize}

\hrule
\vspace{10pt}

\textbf{Team Member 4}
\begin{itemize}
	\item Role(s)
	\begin{enumerate}
		\item Automation Engineer
		\item Performance Tester
	\end{enumerate}
	\item Responsibilities
	\begin{enumerate}
		\item Develops automated test scripts for repetitive testing.
		\item Measures app performance under different conditions to optimize speed and efficiency.
	\end{enumerate}
\end{itemize}

\hrule
\vspace{10pt}

\textbf{Team Member 5}
\begin{itemize}
	\item Role(s)
	\begin{enumerate}
		\item Quality Assurance Engineer
		\item Performance Tester
	\end{enumerate}
	\item Responsibilities
	\begin{enumerate}
		\item Coordinates end-to-end testing and verifies the app's functionality.
		\item Focuses on maintaining consistent performance across updates.
	\end{enumerate}
\end{itemize}

\wss{Your teammates.  Maybe your supervisor.
  You should do more than list names.  You should say what each person's role is
  for the project's verification.  A table is a good way to summarize this information.}

\subsection{SRS Verification Plan}

\wss{List any approaches you intend to use for SRS verification.  This may
  include ad hoc feedback from reviewers, like your classmates (like your
  primary reviewer), or you may plan for something more rigorous/systematic.}

\wss{If you have a supervisor for the project, you shouldn't just say they will
read over the SRS.  You should explain your structured approach to the review.
Will you have a meeting?  What will you present?  What questions will you ask?
Will you give them instructions for a task-based inspection?  Will you use your
issue tracker?}

\wss{Maybe create an SRS checklist?}

\subsubsection{Approaches for SRS Verifiation}

\begin{itemize}
	\item Peer Review Feedback:
	\begin{itemize}
		\item Each team member will review the SRS individually and provide ad hoc feedback. They will focus on identifying ambiguous language, incomplete requirements, and inconsistencies.
		\item Each reviewer will log feedback in a shared document, categorizing issues by priority (e.g., high, medium, low).
	\end{itemize}
	\item Primary Reviewer Session:
	\begin{itemize}
		\item A primary reviewer from the team, ideally someone with experience in requirements verification, will perform a detailed examination. They will document findings, focusing on critical areas such as requirement clarity, feasibility, and completeness.
	\end{itemize}
	\item External Peer Feedback:
	\begin{itemize}
		\item If possible, solicit feedback from classmates or others familiar with requirements engineering. They can offer objective insights and identify issues that may have been overlooked internally.
	\end{itemize}
\end{itemize}

\newpage

\subsubsection{Structured Internal Review Process}

\begin{itemize}
	\item Initial Team Meeting:\
	\begin{itemize}
		\item Hold a structured review meeting where team members present and discuss major sections of the SRS. Key points of focus should include functional requirements, assumptions, and constraints.
		\item Prepare a set of questions to guide the discussion, such as:
		\begin{itemize}
			\item "Are the requirements unambiguous and complete?"
			\item "Is each requirement feasible and realistic?"
			\item "Do the requirements adequately represent the user's needs?"
			\item "Are there any areas that might be challenging to test or verify?"
		\end{itemize}
	\end{itemize}
	\item Task-Based Inspection:
	\begin{itemize}
		\item Team members can take turns as "inspectors," responsible for a task-based evaluation of key areas:
		\begin{itemize}
			\item Ensuring requirements align with project goals.
			\item Verifying each requirement's testability and clarity.
			\item Checking that the document is organized and easy to navigate.
		\end{itemize}
	\end{itemize}
	\item Issue Tracker:
	\begin{itemize}
		\item Use an issue tracker or a collaborative tool to manage findings and resolutions. Each issue should include:
		\begin{itemize}
			\item A summary of the issue.
			\item Priority level (e.g., high, medium, low).
			\item Suggested resolution steps.
			\item A status field for tracking progress until resolved.
		\end{itemize}
	\end{itemize}
\end{itemize}

\newpage

\subsubsection{SRS Quality Checklist for Verification}
\begin{tabular}{|>{\raggedright\arraybackslash}p{3cm}|>{\raggedright\arraybackslash}p{9cm}|>{\raggedright\arraybackslash}p{2cm}|}
	\hline
	\textbf{Checklist Item} & \textbf{Description} & \textbf{Pass/Fail} \\
	\hline
	Requirements Clarity & All requirements are clear, unambiguous, and written in active voice. & \\
	\hline
	Completeness & The SRS includes all functional and non-functional requirements, assumptions, and constraints. & \\
	\hline
	Consistency & Requirements are consistent across the document, with no contradictory statements. & \\
	\hline
	Testability & Each requirement is formulated so that it can be tested through measurable criteria. & \\
	\hline
	Feasibility & All requirements are realistic and achievable within the project's scope and resources. & \\
	\hline
	Traceability & Each requirement is traceable to a higher-level objective or user need. & \\
	\hline
	Maintainability	& The document is well-organized, with clear headings and numbered requirements for easy referencing. & \\
	\hline
	Priority and Dependencies & Requirements are prioritized, and dependencies are explicitly stated where applicable. & \\
	\hline
	User-Centered Language & Requirements reflect user needs and use terminology consistent with user and stakeholder language. & \\
	\hline
	Compliance with Standards & The SRS complies with any applicable standards or best practices in software requirements engineering. & \\
	\hline

\end{tabular}

\newpage

\subsection{Design Verification Plan}

\wss{Plans for design verification}

\wss{The review will include reviews by your classmates}

\wss{Create a checklists?}

\subsubsection{Specification Verification}

\noindent \begin{tabular}{|>{\raggedright\arraybackslash}p{3cm}|>{\raggedright\arraybackslash}p{4cm}|>{\raggedright\arraybackslash}p{3cm}|p{1cm}|p{2.25cm}|}
	\hline
	\textbf{Checklist Items} & \textbf{Description} & \textbf{Verification Method} & \textbf{Pass /Fail} & \textbf{Comments} \\ 
	\hline
	Requirements Completeness & Ensure all system and user requirements are addressed in the design. & Requirements Traceability Matrix & & \\
	\hline
	System Requirements Compliance & Verify that design conforms to software/system specifications. & Specification Review	& & \\
	\hline
	Compliance with Standards & Ensure compliance with coding standards, industry best practices, and regulatory guidelines. & Code Review, Standards Checklist	& & \\
	\hline
	Security Requirements & Confirm that the security design follows the requirements, especially for data privacy and user information. & Security Audit & & \\
	\hline
\end{tabular}

\newpage

\subsubsection{Functional Verification}

\noindent \begin{tabular}{|>{\raggedright\arraybackslash}p{3cm}|>{\raggedright\arraybackslash}p{4cm}|>{\raggedright\arraybackslash}p{3cm}|p{1cm}|p{2.25cm}|}
	\hline
	\textbf{Checklist Items} & \textbf{Description} & \textbf{Verification Method} & \textbf{Pass /Fail} & \textbf{Comments} \\ 
	\hline
	Receipt Scanning Capability	& Verify that receipt scanning feature works as expected with various image qualities. & Functional Testing	& & \\
	\hline
	Expense Categorization Accuracy	& Confirm that expenses are correctly categorized by the AI model. & Model Accuracy Testing	& & \\
	\hline
	User Interface (UI) Responsiveness & Ensure that UI elements are responsive across devices and screen sizes. & UI Testing & & \\
	\hline
	Error Handling and Recovery	& Verify that the app gracefully handles errors, such as scanning failures or incorrect categorizations. & Error Simulation	& & \\
	\hline
	Integration with External Services & Confirm integration with any third-party services (e.g., cloud storage or payment platforms). & Integration Testing & & \\
	\hline
\end{tabular}

\newpage

\subsubsection{Performance Validation}

\noindent \begin{tabular}{|>{\raggedright\arraybackslash}p{3cm}|>{\raggedright\arraybackslash}p{4cm}|>{\raggedright\arraybackslash}p{3cm}|p{1cm}|p{2.25cm}|}
	\hline
	\textbf{Checklist Items} & \textbf{Description} & \textbf{Verification Method} & \textbf{Pass /Fail} & \textbf{Comments} \\ 
	\hline
	System Performance Under Load & Test app performance with simultaneous receipt scans to assess system stability and responsiveness.	& Load Testing & & \\
	\hline
	AI Model Processing Time & Measure the average processing time for receipt categorization. & Model Timing Analysis & & \\
	\hline
	Battery and Resource Consumption & Verify that app resource usage is within acceptable limits to preserve device performance and battery life. & Resource Monitoring & & \\
	\hline
	Network Efficiency & Ensure the app handles network limitations (e.g., slow or intermittent connections) without data loss.	& Network Simulation & & \\
	\hline
	Latency for Real-Time Features & Measure latency in real-time features, ensuring a smooth user experience. & Latency Testing & & \\
	\hline
\end{tabular}

\newpage

\subsubsection{Document Validation}

\noindent \begin{tabular}{|>{\raggedright\arraybackslash}p{3cm}|>{\raggedright\arraybackslash}p{4cm}|>{\raggedright\arraybackslash}p{3cm}|p{1cm}|p{2.25cm}|}
	\hline
	\textbf{Checklist Items} & \textbf{Description} & \textbf{Verification Method} & \textbf{Pass /Fail} & \textbf{Comments} \\ 
	\hline
	User Documentation Completeness	& Verify that user manuals, installation guides, and support documentation are complete and accurate. & Documentation Review & & \\
	\hline
	Developer Documentation Completeness & Ensure all developer-focused documentation (e.g., API docs, setup instructions) is detailed and up to date. & Documentation Review & & \\ 
	\hline
	Code Documentation & Confirm that all code modules and functions are documented according to guidelines. & Code Review & & \\
	\hline
	Test Plans and Results & Verify the completeness of test plans and that test results meet required standards. & Test Report Review & & \\
	\hline
	Change Log and Version History & Ensure that the change log and version history are comprehensive and well-documented. & Change Log Review & & \\ 
	\hline
\end{tabular}

\subsection{Verification and Validation Plan Verification Plan}

\wss{The verification and validation plan is an artifact that should also be
verified.  Techniques for this include review and mutation testing.}

\wss{The review will include reviews by your classmates}

\wss{Create a checklists?}

\subsection{Implementation Verification Plan}

\wss{You should at least point to the tests listed in this document and the unit
  testing plan.}

\wss{In this section you would also give any details of any plans for static
  verification of the implementation.  Potential techniques include code
  walkthroughs, code inspection, static analyzers, etc.}

\wss{The final class presentation in CAS 741 could be used as a code
walkthrough.  There is also a possibility of using the final presentation (in
CAS741) for a partial usability survey.}

\newpage

\subsection{Automated Testing and Verification Tools}

\wss{What tools are you using for automated testing.  Likely a unit testing
  framework and maybe a profiling tool, like ValGrind.  Other possible tools
  include a static analyzer, make, continuous integration tools, test coverage
  tools, etc.  Explain your plans for summarizing code coverage metrics.
  Linters are another important class of tools.  For the programming language
  you select, you should look at the available linters.  There may also be tools
  that verify that coding standards have been respected, like flake9 for
  Python.}

\wss{If you have already done this in the development plan, you can point to
that document.}

\wss{The details of this section will likely evolve as you get closer to the
  implementation.}

\begin{enumerate}
	\item \textbf{Unit Testing Frameworks}
	\begin{itemize}
		\item \textbf{Jest}: Jest is the default testing framework for React Native applications. It supports TypeScript, has a rich API for writing unit tests, and comes with built-in mocking capabilities.
		\item \textbf{React Testing Library}: This library works well with Jest for testing React components, focusing on user interactions and component behavior rather than implementation details.
	\end{itemize}
	\item \textbf{Static Analysis Tools}
	\begin{itemize}
		\item \textbf{ESLint}: ESLint is a powerful linter for JavaScript and TypeScript. It can be configured with TypeScript support to enforce coding standards, detect potential errors, and maintain code quality. It can also be extended with plugins for React and serves as industry best practices for JavaScript/TypeScript related-projects.
		\item \textbf{Prettier}: Integrating Prettier alongside ESLint helps ensure consistent code formatting across the codebase, which helps in improving readability.
	\end{itemize}
	\item \textbf{Continuous Integration (CI) Tools}
	\begin{itemize}
		\item \textbf{GitHub Actions}: This tool can be set up to run workflows that execute tests, linters, and builds automatically on every pull request or commit, ensuring that code quality is maintained consistently (i.e. Husky).
		\item \textbf{Jenkins}: Jenkins is an open-source automation server that facilitates continuous integration and continuous deployment processes. It can be configured to automatically run tests, linting, and builds every time code is pushed to the repository, ensuring that code quality is maintained consistently.
	\end{itemize}
	\item \textbf{Test Coverage Tools}
	\begin{itemize}
		\item \textbf{Istanbul (nyc)}: Istanbul is a code coverage tool that integrates with Jest, providing detailed reports on which parts of the codebase are covered by tests, helping to identify untested code.
		\item \textbf{Codecov or Coveralls}: These services can be used to visualize and summarize code coverage metrics after each build, providing insights into overall coverage and trends over time.
	\end{itemize}
	\item \textbf{Profiling Tools}
	\begin{itemize}
		\item \textbf{React Native Performance Monitor:}: This built-in tool in React Native helps track performance metrics such as frame rates and memory usage, providing insights into the app's performance.
		\item \textbf{Flipper}: A platform for debugging mobile apps, Flipper offers a variety of plugins that assist in profiling and monitoring app performance.
	\end{itemize}
	\item \textbf{Mutation Testing}
	\begin{itemize}
		\item \textbf{Stryker}: Stryker is a mutation testing framework for JavaScript and TypeScript. It can help assess the effectiveness of tests by introducing small changes in the code (mutations) and checking if the tests can catch them.
	\end{itemize}
	\item \textbf{Plans for Summarizing Code Coverage Metrics}
	\begin{itemize}
		\item \textbf{Daily/Weekly Reports:}: The CI pipeline can be configured to generate code coverage reports with every build and aggregate these reports weekly. Tools like Codecov can be used to visualize trends over time.
		\item \textbf{Review Meetings}: Regular review meetings can be conducted to discuss coverage metrics with the team, identifying areas needing improvement and setting goals for coverage percentages.
		\item \textbf{Integrate Coverage Reports in PRs}: Coverage reports should be included in pull request reviews, ensuring that any code changes are evaluated in the context of their impact on overall coverage.
	\end{itemize}
\end{enumerate}

\noindent For additional information, please consult the development plan.
\subsection{Software Validation Plan}

\wss{If there is any external data that can be used for validation, you should
  point to it here.  If there are no plans for validation, you should state that
  here.}

\wss{You might want to use review sessions with the stakeholder to check that
the requirements document captures the right requirements.  Maybe task based
inspection?}

\wss{For those capstone teams with an external supervisor, the Rev 0 demo should 
be used as an opportunity to validate the requirements.  You should plan on 
demonstrating your project to your supervisor shortly after the scheduled Rev 0 demo.  
The feedback from your supervisor will be very useful for improving your project.}

\wss{For teams without an external supervisor, user testing can serve the same purpose 
as a Rev 0 demo for the supervisor.}

\wss{This section might reference back to the SRS verification section.}

\section{System Tests}

\subsection{Tests for Functional Requirements}

The following test cases are divided into subsets corresponding to major functional areas in the application, as detailed in the Software Requirements Specification (SRS) document. 
Each subset addresses a specific functional component, including user account management, receipt scanning and processing, database management, item recognition and categorization, and financial tracking. 
These tests will help verify the functional requirements outlined in the SRS and will help with the creation of those tests to verify the specified functional requirements.

\subsubsection{User Account Management Tests}

This subsection covers tests required to verify tests revolving around user account management. The user account manager allows users to create an account, update it, and log in and out.\\
\textit{Functional Requirements: FR1 - FR5}

\begin{enumerate}

\item{test-UAM-1\\}

\textbf{Description:} The user should be able to create an account with specified name, email address and password

\textbf{Control:} Manual
					
\textbf{Initial State:} The app is on the account creation screen and there is no existing account with the test email
					
\textbf{Input:} Name, valid email, and password
					
\textbf{Output:} Account creation is successful, and the user is redirected to the login screen

\textbf{Test Case Derivation:} This test ensures that the application allows users to create an account. The manager expects a valid name with no numbers or special characters, a valid email (existing email of correct email format) and a password that only contains certain characters to prevent SQL injection
					
\textbf{How test will be performed:} The test will be performed by providing instructions to a user/tester to enter valid name, email and password submissions. They will then determine if the test is successful by verifying that the actual output matches the expected output (as provided in the instructions/referenced in this document)

\item{test-UAM-2\\}

\textbf{Description:} The user should be able to login with valid credentials

\textbf{Control:} Manual
					
\textbf{Initial State:} The app is on the login page and a registered account has been created with the corresponding testing credentials
					
\textbf{Input:} Email and password
					
\textbf{Output:} The app is on the login page and a registered account has been created with the corresponding testing credentials

\textbf{Test Case Derivation:} This test ensures that only authenticated users may access the app's functionalities

\textbf{How test will be performed:} The test will be performed by having a user/tester provide valid credentials to the system and verify that the app redirects them to the home page after successful login

\item{test-UAM-3\\}

\textbf{Description:} The user should be able to logout of the app

\textbf{Control:} Manual
					
\textbf{Initial State:} The user is already logged into the app
					
\textbf{Input:} Logout submission
					
\textbf{Output:} Account logout is successful, and the user is redirected to the login page

\textbf{Test Case Derivation:} This test verifies that the app terminates the user's session and redirects the user to the login page to protect user's account data

\textbf{How test will be performed:} While logged in, a user/tester will click the logout button and observe that the user is redirected to the login screen. Verify that no user-specific data is accessible after logging out

\item{test-UAM-4}

\textbf{Description:} Users should be able to update account information after initial setup

\textbf{Control:} Manual

\textbf{Initial State:} User is logged in and on the account settings screen

\textbf{Input:} Modify fields (e.g., name or password) and save changes

\textbf{Output:} Updated account information is saved and reflected on the profile screen.

\textbf{Test Case Derivation:} The test ensures that users are able to manage and update their account information after initial setup

\textbf{How test will be performed:} A tester/user will modify the account fields, save, and verify that updates are displayed correctly on the profile.
\end{enumerate}

\subsubsection{Image Processing Tests}

This subsection covers tests required to verify tests revolving around receipt scanning and processing. The receipt scanner allows users to upload and preview images\\
\textit{Functional Requirements: FR8 - FR11}

\begin{enumerate}
\item{test-IP-1\\}

\textbf{Description:} User should be able to upload an image for processing

\textbf{Control:} Manual

\textbf{Initial State:} The application is on the receipt upload screen.

\textbf{Input:} Select and upload a receipt image from the device's file storage

\textbf{Output:} The uploaded image is successfully displayed for preview

\textbf{Test Case Derivation:} The test verifies that users are able to upload images from their device and preview them before receipt scanning

\textbf{How test will be performed:} The user/tester will select an image from storage, upload it, and confirm the image preview correctly displayed

\item{test-IP-2\\}

\textbf{Description:} Users should be able to preview an uploaded image before processing and confirm/reupload the image based on preference

\textbf{Control:} Manual

\textbf{Initial State:} Image is uploaded and displayed as a preview.

\textbf{Input:} Confirm or reupload the previewed image

\textbf{Output:} Confirmed image proceeds to processing, or retake restarts the upload process

\textbf{Test Case Derivation:} Ensures users can review and confirm the uploaded image, as required for accuracy before processing.

\textbf{How test will be performed:} The user/tester previews the image, confirms, and observes if it advances to processing. Alternatively, users can retake and verify they are taken back to the image upload process
\end{enumerate}

\subsubsection{Manual Expense Input Tests}

This subsection covers tests that verify functionality for allowing users to manually input their expenses in the case that their receipt is unable to be processed. 
The manual input system is responsible for outlining fields that the user must fill out for their expense(s) to be tracked.\\
\textit{Functional Requirements: FR12-13}

\begin{enumerate}
\item{test-MIS-1\\}

\textbf{Description:} Users should be able to manually input expenses

\textbf{Control:} Manual

\textbf{Initial State:} Image captured or uploaded cannot be processed or the user wants to manually input items from their receipt from the expense tracker view

\textbf{Input:} Entered items, item costs, item quantities, categories and receipt date

\textbf{Output:} Items, quantities, costs, categories and receipt date are correctly processed and displayed 

\textbf{Test Case Derivation:} Ensures users can manually input their receipt/expense in case their image isn't able to be processed or they want to manually input them. The test verifies that manually inputted expense data is processed correctly

\textbf{How test will be performed:} From the expense tracker page, the user/tester will select the “manually input expenses” option and fill out the corresponding fields for category, item, item cost and date. They will then validate that all manual entries have been processed in the finalized expenses screen and match their inputs.	

\end{enumerate}

\subsubsection{Database Management Tests}

This subsection covers tests required to verify functionality for database management. The database managers securely stores user account information and user receipt information.\\
\textit{Functional Requirements: FR14 - FR15}

\begin{enumerate}
\item{test-DM-1\\}

\textbf{Description:} Account information should be protected and securely stored in the database

\textbf{Control:} Automatic

\textbf{Initial State:} Application database is empty or has only encrypted data.

\textbf{Input:} Create a new user account.

\textbf{Output:} User data is stored securely in the database, encrypted.

\textbf{Test Case Derivation:} The test validates that user account information is protected and verifies that the application securely handles user data

\textbf{How test will be performed:} Check database entries post-account creation to verify data encryption and account information

\item{test-DM-2\\}

\textbf{Description:} Receipt data must be securely stored in database

\textbf{Control:} Automatic

\textbf{Initial State:} Application database has no receipt data for the test user.

\textbf{Input:} Upload a receipt image.

\textbf{Output:} Receipt image and extracted data are securely stored in the database.

\textbf{Test Case Derivation:} The test verifies receipt data is securely stored, ensuring data privacy

\textbf{How test will be performed:} Check database entries post-upload to confirm storage security and receipt information
\end{enumerate}

\subsubsection{Item Recognition and Categorization}

This subsection covers tests required to verify functionality for item recognition and categorization from receipt scanning. The receipt scanner is responsible for recognizing and categorizing items from users' uploaded receipts.\\
\textit{Functional Requirements: FR16 - FR18}

\begin{enumerate}
\item{test-RS-1\\}

\textbf{Description:} Uploaded receipts should be processed and all items, their quantities and prices should all be correctly recognized with an accuracy of TARGET\_OCR\_MODEL\_ACCURACY

\textbf{Control:} Automatic

nitial State: User confirms uploaded receipt is ready to be processed

\textbf{Input:} Run item recognition on a sample receipt

\textbf{Output:} Recognized items, quantities, and prices are displayed accurately

\textbf{Test Case Derivation:} The test verifies the accuracy of the OCR model by identifying items on the inputted receipt

\textbf{How test will be performed:} Process the receipt image and verify recognized details against the original receipt

\item{test-RS-2\\}

\textbf{Description:} Uploaded receipts should categorize items on them under common expense categories (i.e. groceries, entertainment, etc..)

\textbf{Control:} Automatic

\textbf{Initial State:} Receipt has been scanned and items are recognized from the receipt

\textbf{Input:} Apply categorization based on item names.

\textbf{Output:} Items are correctly categorized (i.e. milk under “Groceries”).

\textbf{Test Case Derivation:} The test validates organized expense tracking accuracy by categorizing items under common expense categories

\textbf{How test will be performed:} Run categorization on a processed receipt and check that items are correctly sorted
\end{enumerate}

\subsubsection{Financial Tracking Tests}

This subsection verifies the functionality of financial tracking features such has spending history and budget management. The financial tracker is responsible for storing users expenses/spendings, as well as allowing users to set and track budgets.\\
\textit{Functional Requirements: 19-21}

\begin{enumerate}

\item{test-FT-1\\}

\textbf{Description:} Users should be able view spending their spending history by category and trends

\textbf{Control:} Manual

\textbf{Initial State:} Application has recorded user's past expenses and user is on the expense tracking view

\textbf{Input:} Generate spending history overview

\textbf{Output:} Accurate summary of total spending by category and time period

\textbf{Test Case Derivation:} The test verifies users can monitor their budgets through viewing their spending trends

\textbf{How test will be performed:} The tester/user should be able to trigger spending history generation and verify summaries match expected data upon entering the expense tracker view

\item{test-FT-2\\}

\textbf{Description:} Users should be able to set and track their budgets

\textbf{Control:} Manual

\textbf{Initial State:} User is on the budgeting window and no current budget is set for the specified category

\textbf{Input:} Budget limit for a specific category

\textbf{Output:} Budget limit is saved and displayed in tracking

\textbf{Test Case Derivation:} This test confirms the ability for users to set and track budgets

\textbf{How test will be performed:} The tester/user sets a budget for a specific category, then verify it appears and updates as expected in tracking

\end{enumerate}

\subsection{Tests for Nonfunctional Requirements}

\wss{The nonfunctional requirements for accuracy will likely just reference the
  appropriate functional tests from above.  The test cases should mention
  reporting the relative error for these tests.  Not all projects will
  necessarily have nonfunctional requirements related to accuracy.}

\wss{For some nonfunctional tests, you won't be setting a target threshold for
passing the test, but rather describing the experiment you will do to measure
the quality for different inputs.  For instance, you could measure speed versus
the problem size.  The output of the test isn't pass/fail, but rather a summary
table or graph.}

\wss{Tests related to usability could include conducting a usability test and
  survey.  The survey will be in the Appendix.}

\wss{Static tests, review, inspections, and walkthroughs, will not follow the
format for the tests given below.}

\wss{If you introduce static tests in your plan, you need to provide details.
How will they be done?  In cases like code (or document) walkthroughs, who will
be involved? Be specific.}

\subsubsection{Area of Testing1}
		
\paragraph{Title for Test}

\begin{enumerate}

\item{test-id1\\}

Type: Functional, Dynamic, Manual, Static etc.
					
\textbf{Initial State:} 
					
Input/Condition: 
					
Output/Result: 
					
\textbf{How test will be performed:} 
					
\item{test-id2\\}

Type: Functional, Dynamic, Manual, Static etc.
					
\textbf{Initial State:} 
					
\textbf{Input:} 
					
\textbf{Output:} 
					
\textbf{How test will be performed:} 

\end{enumerate}

\subsubsection{Area of Testing2}

...

\subsection{Traceability Between Test Cases and Requirements}

\wss{Provide a table that shows which test cases are supporting which
  requirements.}

\section{Unit Test Description}

\wss{This section should not be filled in until after the MIS (detailed design
  document) has been completed.}

\wss{Reference your MIS (detailed design document) and explain your overall
philosophy for test case selection.}  

\wss{To save space and time, it may be an option to provide less detail in this section.  
For the unit tests you can potentially layout your testing strategy here.  That is, you 
can explain how tests will be selected for each module.  For instance, your test building 
approach could be test cases for each access program, including one test for normal behaviour 
and as many tests as needed for edge cases.  Rather than create the details of the input 
and output here, you could point to the unit testing code.  For this to work, you code 
needs to be well-documented, with meaningful names for all of the tests.}

\subsection{Unit Testing Scope}

\wss{What modules are outside of the scope.  If there are modules that are
  developed by someone else, then you would say here if you aren't planning on
  verifying them.  There may also be modules that are part of your software, but
  have a lower priority for verification than others.  If this is the case,
  explain your rationale for the ranking of module importance.}

\subsection{Tests for Functional Requirements}

\wss{Most of the verification will be through automated unit testing.  If
  appropriate specific modules can be verified by a non-testing based
  technique.  That can also be documented in this section.}

\subsubsection{Module 1}

\wss{Include a blurb here to explain why the subsections below cover the module.
  References to the MIS would be good.  You will want tests from a black box
  perspective and from a white box perspective.  Explain to the reader how the
  tests were selected.}

\begin{enumerate}

\item{test-id1\\}

Type: \wss{Functional, Dynamic, Manual, Automatic, Static etc. Most will
  be automatic}
					
\textbf{Initial State:} 
					
\textbf{Input:} 
					
\textbf{Output:} \wss{The expected result for the given inputs}

\textbf{Test Case Derivation:} \wss{Justify the expected value given in the Output field}

\textbf{How test will be performed:} 
					
\item{test-id2\\}

Type: \wss{Functional, Dynamic, Manual, Automatic, Static etc. Most will
  be automatic}
					
\textbf{Initial State:} 
					
\textbf{Input:} 
					
\textbf{Output:} \wss{The expected result for the given inputs}

\textbf{Test Case Derivation:} \wss{Justify the expected value given in the Output field}

\textbf{How test will be performed:} 

\item{...\\}
    
\end{enumerate}

\subsubsection{Module 2}

...

\subsection{Tests for Nonfunctional Requirements}

\wss{If there is a module that needs to be independently assessed for
  performance, those test cases can go here.  In some projects, planning for
  nonfunctional tests of units will not be that relevant.}

\wss{These tests may involve collecting performance data from previously
  mentioned functional tests.}

\subsubsection{Module ?}
		
\begin{enumerate}

\item{test-id1\\}

Type: \wss{Functional, Dynamic, Manual, Automatic, Static etc. Most will
  be automatic}
					
\textbf{Initial State:} 
					
Input/Condition: 
					
Output/Result: 
					
\textbf{How test will be performed:} 
					
\item{test-id2\\}

Type: Functional, Dynamic, Manual, Static etc.
					
\textbf{Initial State:} 
					
\textbf{Input:} 
					
\textbf{Output:} 
					
\textbf{How test will be performed:} 

\end{enumerate}

\subsubsection{Module ?}

...

\subsection{Traceability Between Test Cases and Modules}

\wss{Provide evidence that all of the modules have been considered.}
				
\bibliographystyle{plainnat}

\bibliography{../../refs/References}

\newpage

\section{Appendix}

This is where you can place additional information.

\subsection{Symbolic Parameters}

The definition of the test cases will call for SYMBOLIC\_CONSTANTS.
Their values are defined in this section for easy maintenance.

\subsection{Usability Survey Questions?}

\wss{This is a section that would be appropriate for some projects.}

\newpage{}
\section*{Appendix --- Reflection}

\wss{This section is not required for CAS 741}

The information in this section will be used to evaluate the team members on the
graduate attribute of Lifelong Learning.

\begin {enumerate}
\item \textbf{Why is it important to create a development plan prior to starting the
Project?}

Creating a development plan before starting the project is crucial so the whole team can discuss/agree upon the key project details and scope. It is vital to lay out the groundwork for our project and define the direction needed to achieve our goals optimally. We found it especially important to discuss team dynamics – specifying meeting details, expectations from each member of the team, and the general workflow plan. Defining these elements upfront helps keep everyone aligned and organized before we dive into the project details and implementation. Breaking down the project into high-level milestones allows us to create well-defined and achievable timelines to guide our progress.

\item \textbf{In your opinion, what are the advantages and disadvantages of using CI/CD?}

We believe that CI/CD is a great tool due to it allowing automation and quality control within our development workflow. This significantly reduces manual testing labour and saves time which will be crucial for the development of the Plutos app. We aim to at least include running tests, linters, and formatters within our CI pipeline so that we may be confident that all changes made meet a certain quality level. One drawback that we will need to be aware of is that the setup of the CI/CD pipeline may pose a challenge due to the team’s lack of experience in setting up such a workflow. Having too much automation could lead us to have false confidence in our code, especially if our test coverage is not meeting standards due to rapid development. It’s important not to rely too heavily on the CI/CD pipeline and understand that it’s a tool that is meant to help gauge the overall quality of our code and not something that will replace manual testing.

\item \textbf{What disagreements did your group have in this deliverable, if any, and how did you resolve them?}

Our team had differing opinions when we discussed the timeline of our project. Some members felt we could reduce the time allocated for end-to-end testing from five weeks to three, while others preferred to extend the frontend and backend development by an additional week. We also had discussions for how much buffer time was needed to account for potential delays. After some discussion, we came to an agreement that the frontend and backend could be done in parallel, allowing us to dedicate 6 weeks to both sections. These differences were ultimately resolved through mutual understanding, as many of us would be unavailable during exam season in December and January. What ultimately helped us resolve our differences was the fact that we all remained realistic about potential challenges we might face in the future.

\end{enumerate}

\begin{enumerate}
  \item What went well while writing this deliverable? 
  \item What pain points did you experience during this deliverable, and how
    did you resolve them?
  \item What knowledge and skills will the team collectively need to acquire to
  successfully complete the verification and validation of your project?
  Examples of possible knowledge and skills include dynamic testing knowledge,
  static testing knowledge, specific tool usage, Valgrind etc.  You should look to
  identify at least one item for each team member.
  \item For each of the knowledge areas and skills identified in the previous
  question, what are at least two approaches to acquiring the knowledge or
  mastering the skill?  Of the identified approaches, which will each team
  member pursue, and why did they make this choice?
\end{enumerate}

\end{document}