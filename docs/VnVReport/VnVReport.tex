\documentclass[12pt, titlepage]{article}

\usepackage{booktabs}
\usepackage{tabularx}
\usepackage{hyperref}
\hypersetup{
    colorlinks,
    citecolor=black,
    filecolor=black,
    linkcolor=red,
    urlcolor=blue
}
\usepackage[round]{natbib}
\usepackage{longtable}
\input{../Comments}
%% Common Parts

\newcommand{\progname}{Plutos} % PUT YOUR PROGRAM NAME HERE
\newcommand{\authname}{Team \#10, Plutos
\\ Jason Tan
\\ Payton Chan
\\ Eric Chen
\\ Fondson Lu
\\ Angela Wang} % AUTHOR NAMES                  

\usepackage{hyperref}
    \hypersetup{colorlinks=true, linkcolor=blue, citecolor=blue, filecolor=blue,
                urlcolor=blue, unicode=false}
    \urlstyle{same}
                                


\begin{document}

\title{Verification and Validation Report: \progname} 
\author{\authname}
\date{\today}
	
\maketitle

\pagenumbering{roman}

\section{Revision History}

\begin{tabularx}{\textwidth}{p{3cm}p{2cm}X}
\toprule {\bf Date} & {\bf Version} & {\bf Notes}\\
\midrule
Date 1 & 1.0 & Notes\\
Date 2 & 1.1 & Notes\\
\bottomrule
\end{tabularx}

~\newpage

\section{Symbols, Abbreviations and Acronyms}


Refer to Section 1.3 of the
\href{https://github.com/PlutosCapstone/Plutos/blob/main/docs/SRS/SRS.pdf}{Software
Requirements Specification (SRS)} document for the list of symbols,
abbreviations, and acronyms.


In addition, the following abbreviations are used in this document:\\

\renewcommand{\arraystretch}{1.2}
\begin{table}[h!]
\caption{Symbols, Abbreviations, and Acronyms}
\begin{tabularx}{\textwidth}{l l}
  \toprule		
  \textbf{symbol} & \textbf{description}\\
  \midrule 
  V\&V & Verification and Validation\\
  UI & User Interface\\
  OCR & Optical Character Recognition\\
  SQL & Structured Query Language\\
  GDPR & General Data Protection Regulation\\
  \bottomrule
\end{tabularx}
\end{table}

\newpage

\tableofcontents

\listoftables %if appropriate

\listoffigures %if appropriate

\newpage

\pagenumbering{arabic}

This document reports the results of the Verification and Validation (V\&V)
process for the \progname software. The V\&V plan is documented in the
\href{https://github.com/PlutosCapstone/Plutos/blob/main/docs/VnVPlan/VnVPlan.pdf}{Verification
and Validation Plan} document. 

\section{Functional Requirements Evaluation}

The functional system tests can be found in Section 4.1 of the
\href{https://github.com/PlutosCapstone/Plutos/blob/main/docs/VnVPlan/VnVPlan.pdf}{Verification
and Validation Plan} document. These tests are all performed manually.

\begin{table}[h!]
\centering
\caption{Functional Requirements Evaluation}
\begin{tabularx}{\textwidth}{>{\centering\arraybackslash}X >{\centering\arraybackslash}X >{\centering\arraybackslash}X}
  \toprule
  \textbf{Test ID} & \textbf{Pass/Fail} & \textbf{Comments} \\
  \midrule
  test-UAM-1 & Pass &  \\
  test-UAM-2 & Pass &  \\
  test-UAM-3 & Pass &  \\
  test-UAM-4 & Fail & Not yet implemented \\
  test-UAM-5 & Pass &  \\
  test-UAM-6 & Pass &  \\
  \midrule
  test-IP-1 & Pass & \\
  test-IP-2 & Pass & \\
  test-IP-3 & Fail & Not yet implemented \\
  \midrule
  test-MIS-1 & Pass & \\
  \midrule
  test-DM-1 & Pass & \\
  test-DM-2 & Pass & \\
  \midrule
  test-RS-1 & Pass & \\
  test-RS-2 & Pass & \\
  test-RS-3 & Pass & \\
  \midrule
  test-FT-1 & Fail & Not yet implemented \\
  test-FT-2 & Pass & \\
  test-FT-3 & Fail & Not yet implemented \\
  \bottomrule
\end{tabularx}
\end{table}


\section{Nonfunctional Requirements Evaluation}

The nonfunctional system tests can be found in Section 4.2 of the
\href{https://github.com/PlutosCapstone/Plutos/blob/main/docs/VnVPlan/VnVPlan.pdf}{Verification
and Validation Plan} document. Tests without comments are performed as described
in the plan.


\begin{longtable}{>{\centering\arraybackslash}p{0.2\textwidth} >{\centering\arraybackslash}p{0.2\textwidth} >{\centering\arraybackslash}p{0.5\textwidth}}
  \caption{Nonfunctional Requirements Evaluation}\\
    \toprule
    \textbf{Test ID} & \textbf{Pass/Fail} & \textbf{Comments} \\
    \midrule
    test-ACC-1 & Pass &
    \href{https://github.com/PlutosCapstone/Plutos/tree/main/src/server/tests/imageProcessing/data/categorization/receipt_items_output.csv}{Actual
    output}; accuracy is 47/57 = 82.46\%, which meets the threshold of 80\%. \\
    test-ACC-2 & Pass & By manually comparing the input
    \href{https://github.com/PlutosCapstone/Plutos/tree/main/src/server/tests/imageProcessing/data/parsing/input}{
    (set of receipt images)} with the
    \href{https://github.com/PlutosCapstone/Plutos/tree/main/src/server/tests/imageProcessing/data/parsing/input}{resulting
    output}, and calculating the accuracy as descripted in the V\&V Plan,
    current accuracy is ~80\%, which meets the threshold of 80\%.
    \begin{itemize}
      \item \textit{foodbasics\_1.jpg}: 13.5/15 = 90\%
      \item \textit{foodbasics\_2.jpg}: 7/9 = 77.78\%
      \item \textit{walmart\_1.jpg}: 7/10 = 70\%
      \item \textit{costco\_1.jpg}: 18.5/23 = 76.09\%
      \item Overall accuracy: 46/57 = 80.70\%
    \end{itemize}\\
    test-ACC-3 & Pass &  \\
    test-ACC-4 & Pass &  \\
    test-ACC-5 & Pass &  \\
    \midrule
    test-PERF-1 & Pass &  \\
    test-PERF-2 & Pass &  \\
    test-PERF-3 & Fail & Load testing has not yet been performed \\
    \midrule
    test-USAB-1 & Pass &  \\
    test-USAB-2 & Pass &  \\
    test-USAB-3 & Pass &  \\
    test-USAB-4 & Pass &  \\
    \midrule
    test-SEC-1 & Pass &  \\
    \midrule
    test-MTB-1 & Pass & System stability has been tested, but application is not
    backward compatible since it is still under active development. \\
    test-MTB-2 & Pass &  \\
    test-MTB-3 & Pass &  \\
    \midrule
    test-PORT-1 & Pass &  \\
    test-PORT-2 & Pass &  \\
    test-PORT-3 & Pass &  \\
    \midrule
    test-REUS-1 & Pass &  \\
    test-REUS-2 & Pass &  \\
    \midrule
    test-UND-1 & Pass &  \\
    test-UND-2 & Pass &  \\
    test-UND-3 & Pass &  \\
    \midrule
    test-LEGAL-1 & Fail & The application is still under active development, so
    it is still using the testing environment and not all security features are
    active. \\
    \bottomrule
\end{longtable}


	
\section{Comparison to Existing Implementation}	

This section will not be appropriate for every project.

\section{Unit Testing}

\subsection{Front-end unit tests}

All front-end unit tests can be found in the \href{https://github.com/PlutosCapstone/Plutos/tree/main/src/client/tests}{test directory}\\
Refer to Table \ref{tab:unit-testing} for unit test traceability table

\begin{table}[h]
  \centering
  \renewcommand{\arraystretch}{1.3}
  \begin{tabular}{| m{5cm} | m{8cm} |}
      \hline
      \textbf{Test} & \textbf{Testing plan} \\
      \hline
      test-UAM-1: Account creation &  \\
      \hline
      test-UAM-2: User login &  \\
      \hline
      test-UAM-3: User logout &  \\
      \hline
      test-UAM-4: Account update &  \\
      \hline
      test-UAM-5: Authorization access &  \\
      \hline
      test-UAM-6: Password reset & Manual testing \\
      \hline
      test-IP-1: Image upload & Manual testing \\
      \hline
      test-IP-2: Image preview &  \\
      \hline
      test-IP-3: Image upload file size limit &  \\
      \hline
      test-MIS-1: Manual input expense & AddExpenseView.test.tsx, AddExpenseModal.test.tsx \\
      \hline
      test-FT-1: View spending history and trends & ExpensesList.test.tsx, HomePageMetricsBox.test.tsx, SpendingDetails.test.tsx \\
      \hline
      test-FT-2: Set and track budget & BudgetBoxDetails.test.tsx, MyBudgetsBox.test.tsx, NewBudgetModal.test.tsx \\
      \hline
      test-FT-3: Notification when user approaching limit & Not implemented \\
      \hline
  \end{tabular}
  \caption{Unit Testing Table} \label{tab:unit-testing}
\end{table}


\section{Changes Due to Testing}

\wss{This section should highlight how feedback from the users and from 
the supervisor (when one exists) shaped the final product.  In particular 
the feedback from the Rev 0 demo to the supervisor (or to potential users) 
should be highlighted.}

\section{Automated Testing}
		
\section{Trace to Requirements}
		
\section{Trace to Modules}		

\section{Code Coverage Metrics}

\bibliographystyle{plainnat}
\bibliography{../../refs/References}

\newpage{}
\section*{Appendix --- Reflection}

The information in this section will be used to evaluate the team members on the
graduate attribute of Reflection.

\begin {enumerate}
\item \textbf{Why is it important to create a development plan prior to starting the
Project?}

Creating a development plan before starting the project is crucial so the whole team can discuss/agree upon the key project details and scope. It is vital to lay out the groundwork for our project and define the direction needed to achieve our goals optimally. We found it especially important to discuss team dynamics – specifying meeting details, expectations from each member of the team, and the general workflow plan. Defining these elements upfront helps keep everyone aligned and organized before we dive into the project details and implementation. Breaking down the project into high-level milestones allows us to create well-defined and achievable timelines to guide our progress.

\item \textbf{In your opinion, what are the advantages and disadvantages of using CI/CD?}

We believe that CI/CD is a great tool due to it allowing automation and quality control within our development workflow. This significantly reduces manual testing labour and saves time which will be crucial for the development of the Plutos app. We aim to at least include running tests, linters, and formatters within our CI pipeline so that we may be confident that all changes made meet a certain quality level. One drawback that we will need to be aware of is that the setup of the CI/CD pipeline may pose a challenge due to the team’s lack of experience in setting up such a workflow. Having too much automation could lead us to have false confidence in our code, especially if our test coverage is not meeting standards due to rapid development. It’s important not to rely too heavily on the CI/CD pipeline and understand that it’s a tool that is meant to help gauge the overall quality of our code and not something that will replace manual testing.

\item \textbf{What disagreements did your group have in this deliverable, if any, and how did you resolve them?}

Our team had differing opinions when we discussed the timeline of our project. Some members felt we could reduce the time allocated for end-to-end testing from five weeks to three, while others preferred to extend the frontend and backend development by an additional week. We also had discussions for how much buffer time was needed to account for potential delays. After some discussion, we came to an agreement that the frontend and backend could be done in parallel, allowing us to dedicate 6 weeks to both sections. These differences were ultimately resolved through mutual understanding, as many of us would be unavailable during exam season in December and January. What ultimately helped us resolve our differences was the fact that we all remained realistic about potential challenges we might face in the future.

\end{enumerate}

\begin{enumerate}
  \item What went well while writing this deliverable? 
  \item What pain points did you experience during this deliverable, and how
    did you resolve them?
  \item Which parts of this document stemmed from speaking to your client(s) or
  a proxy (e.g. your peers)? Which ones were not, and why?
  \item In what ways was the Verification and Validation (VnV) Plan different
  from the activities that were actually conducted for VnV?  If there were
  differences, what changes required the modification in the plan?  Why did
  these changes occur?  Would you be able to anticipate these changes in future
  projects?  If there weren't any differences, how was your team able to clearly
  predict a feasible amount of effort and the right tasks needed to build the
  evidence that demonstrates the required quality?  (It is expected that most
  teams will have had to deviate from their original VnV Plan.)
\end{enumerate}

\end{document}