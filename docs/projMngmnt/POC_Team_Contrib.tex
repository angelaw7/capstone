\documentclass{article}

\usepackage{float}
\restylefloat{table}

\usepackage{booktabs}

\title{Team Contributions: POC\\\progname}

\author{\authname}

\date{}

\input{../Comments}
%% Common Parts

\newcommand{\progname}{Plutos} % PUT YOUR PROGRAM NAME HERE
\newcommand{\authname}{Team \#10, Plutos
\\ Jason Tan
\\ Payton Chan
\\ Eric Chen
\\ Fondson Lu
\\ Angela Wang} % AUTHOR NAMES                  

\usepackage{hyperref}
    \hypersetup{colorlinks=true, linkcolor=blue, citecolor=blue, filecolor=blue,
                urlcolor=blue, unicode=false}
    \urlstyle{same}
                                


\begin{document}

\maketitle

This document summarizes the contributions of each team member up to the POC
Demo.  The time period of interest is the time between the beginning of the term
and the POC demo.

\section{Demo Plans}

We will be demonstrating a working prototype of the Plutos application, with the
following base functionalities:
\begin{itemize}
    \item Upload a receipt image
    \item Pass the image through the machine learning (ML) model to extract
    information and categorize items
    \item Display the extracted information to the screen
\end{itemize}

These features would serve as the base for the product and to demonstrate the
feasibility of the project.

\section{Team Meeting Attendance}

\begin{table}[H]
\centering
\begin{tabular}{ll}
\toprule
\textbf{Student} & \textbf{Meetings}\\
\midrule
\textbf{Total} & \textbf{1}\\
Payton Chan & 1\\
Eric Chen & 1\\
Fondson Lu & 1\\
Jason Tan & 1\\
Angela Wang & 1\\
\bottomrule
\end{tabular}
\end{table}

We only had \href{https://github.com/PlutosCapstone/Plutos/issues/9}{one team
meeting} to brainstorm the project idea. All other communication has been done
asynchronously through our Discord server or informally when we saw each other
during the week.

\section{Supervisor/Stakeholder Meeting Attendance}

\begin{table}[H]
\centering
\begin{tabular}{ll}
\toprule
\textbf{Student} & \textbf{Meetings}\\
\midrule
\textbf{Total} & \textbf{0}\\
Payton Chan & 0\\
Eric Chen & 0\\
Fondson Lu & 0\\
Jason Tan & 0\\
Angela Wang & 0\\
\bottomrule
\end{tabular}
\end{table}

We do not have a supervisor for our project. Additionally, we conducted a survey
for requirements elicitation, which did not require any direct meetings with
stakeholders. 

\section{Lecture Attendance}

\begin{table}[H]
\centering
\begin{tabular}{ll}
\toprule
\textbf{Student} & \textbf{Lectures}\\
\midrule
\textbf{Total} & \textbf{6}\\
Payton Chan & 6\\
Eric Chen & 6\\
Fondson Lu & 6\\
Jason Tan & 6\\
Angela Wang & 6\\
\bottomrule
\end{tabular}
\end{table}


\section{TA Document Discussion Attendance}

\begin{table}[H]
\centering
\begin{tabular}{ll}
\toprule
\textbf{Student} & \textbf{Lectures}\\
\midrule
\textbf{Total} & \textbf{3}\\
Payton Chan & 3\\
Eric Chen & 3\\
Fondson Lu & 3\\
Jason Tan & 3\\
Angela Wang & 3\\
\bottomrule
\end{tabular}
\end{table}

TA Document Discussions:
\begin{itemize}
    \item 10/07/2024 -- Software Requirements Specification (SRS) Document
    \item 10/21/2024 -- Hazard Analysis Document
    \item 10/28/2024 -- Verification and Validation Plan (VnV) Document
\end{itemize}
\section{Commits}

\wss{For each team member how many commits to the main branch have been made
over the time period of interest.  The total is the total number of commits for
the entire team since the beginning of the term.  The percentage is the
percentage of the total commits made by each team member.}

\begin{table}[H]
\centering
\begin{tabular}{lll}
\toprule
\textbf{Student} & \textbf{Commits} & \textbf{Percent}\\
\midrule
Total & Num & 100\% \\
Name 1 & Num & \% \\
Name 2 & Num & \% \\
Name 3 & Num & \% \\
Name 4 & Num & \% \\
Name 5 & Num & \% \\
\bottomrule
\end{tabular}
\end{table}

\wss{If needed, an explanation for the counts can be provided here.  For
instance, if a team member has more commits to unmerged branches, these numbers
can be provided here.  If multiple people contribute to a commit, git allows for
multi-author commits.}

\section{Issue Tracker}

\wss{For each team member how many issues have they authored (including open and
closed issues (O+C)) and how many have they been assigned (only counting closed
issues (C only)) over the time period of interest.}

\begin{table}[H]
\centering
\begin{tabular}{lll}
\toprule
\textbf{Student} & \textbf{Authored (O+C)} & \textbf{Assigned (C only)}\\
\midrule
Name 1 & Num & Num \\
Name 2 & Num & Num \\
Name 3 & Num & Num \\
Name 4 & Num & Num \\
Name 5 & Num & Num \\
\bottomrule
\end{tabular}
\end{table}

\wss{If needed, an explanation for the counts can be provided here.}

\section{CICD}

\wss{Say how CICD will be used in your project}

\wss{If your team has additional metrics of productivity, please feel free to
add them to this report.}

\end{document}